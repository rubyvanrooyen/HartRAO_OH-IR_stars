\section{Analysis}
\label{sec:analysis}

Time series analysis of astronomy data is slightly more challenging because the data tends to be stochastic, non-stationary and irregularly sampled.
Making it an uncomfortable fit for standard signal processing methods.
% Consequently, this generally requires some form of up-sampling through interpolation methods, as well as data conditioning such as normalisation.
To determine the phase lag and coherency of the HartRAO OH monitoring data, appropriate methods are selected to deal with the idiosyncrasies of the observed data.

The HartRAO OH time series data monitored pulsed radiation from OH/IR stars over a multi-year campaign, providing good quality, frequently sampled observational data.
The ligth curves contains multiple cycles of periodic signals, and are well suited for analysis using the Lomb-Scargle periodogram
\citep{2018ApJS..236...16V}.

In order to robustify the numerical calculations, some pre-processing of the data was done before analysis.
This was limited to the manual removal of excessive outlier data points, commonly associated with GLONASS interference, as discussed in \cref{sec:observations}.
Given the high sample rate and length of the HartRAO time series data, excessive smoothing is not required and removal of extreme outliers is generally sufficient.
More automated cleaning can also be accomplished using signal processing algorithms such as the Savitzky Golay smoothing filter
\citep{1964AnaCh..36.1627S}.
A detailed discussion of this scheme is provided in \cref{app:cleaning} applied to the maximum light curve for IK Tau (\cref{fig:iktaublueclean}).
Since the HartRAO OH maser monitoring program spanned several years, some time series data may require removing long term trends introducing a bias over time.
This is easily achieved by fitting and subtracting a low order polynomial to the time series.

To calculate the phase difference (lag) between two signals, an estimate of the frequency (or period) of the signals is obtained.
The Lomb-Scargle periodogram is widely used in the analysis of the light curves of variable stars
\citep{1990ApJ...348..700K}.
In essence, the ``fast Lomb-Scargle algorithm'' computes a Fourier-like power spectrum based on the NFFT algorithm that provides an estimate of the period of oscillation for the variable time series
\citep{1989ApJ...338..277P,2018ApJS..236...16V}.
Only a few oscillations are required for computation and the computational effort is similar.
The Lomb-Scargle periodogram, selects a suitable frequency grid automatically based on the input data.
It should be noted that sometimes irregularly-sampled data can be severely undersampled, requiring the user to specify a higher Nyquist frequency when computing the periodogram
\citep{2006MNRAS.371.1390K}.

Once a period is obtained, the time series can be folded by plotting the periodogram as a function of phase rather than time.
This superimposes the periodic cycles over each other, clearly showing the shape of a cycle.
The HartRAO OH monitoring data is very well sampled and do not require folding to extract the shape of the periodic cycles.
% It does show the distribution of frequency and amplitude of the cycles over the multi-year monitoring period.