\section{Observations}
\label{sec:observations}

A set of 20 OH/IR stars was selected for monitoring in the ground state OH lines using the 26--m telescope at Hartebeesthoek
\citep{West98}.
They were monitored from late 1985 until early 1996.
Initially observations were made every two weeks, but the rapid increases in brightness for some of the sources were found to be undersampled and as a result the observing frequency was increased to once a week in mid 1993.
GLONASS interference affected the data
\citep{ComWestGay94}
and this eventually led to the termination of the programme.

The HartRAO antenna had an aperture efficiency of 0.51 when the subreflector was tilted for observing with the off-axis 18cm feed horn.
The beamwidth was $0.5^\circ$ and the pointing accuracy was $30''$.
The spectrometer accepted only one polarization, and left circular
polarization was normally used.  The zenith system temperature was typically 40--50 K.
A correlator bandwidth of 0.64--MHz, covered by 256 channels, provided a velocity resolution of 0.45 km s$^{-1}$.
Most sources were observed at 1612--MHz, although some were also observed at 1665-- and 1667--MHz.
The {\em rms\/} noise on a single fifteen minute spectrum was 0.6 Jy.
Standard continuum sources
\citep{Ott94}
and the 1667--MHz OH absorption towards W12 were used for calibration.

Baseline curvature in the spectra was removed by subtracting low-order polynomials fitted to signal-free segments.
Atmospheric absorption was corrected using a formula obtained from the Jodrell Bank Observatory.
Data affected by rain and the proximity of the Sun were discarded.
After correction, spectra taken in any given 24--hour period were averaged.
Time series were created for all channels with a useful signal to noise ratio.

Interference from GLONASS satellites increased the noise level in the spectra, suppressed the maser lines, affected the system temperature calibration and introduced excessive curvature into the spectrum baselines
\citep{ComWestGay94}.
Manual correction of GLONASS interference
\citep{West98}
was the method of choice for the sources presented in the paper.
The maximum and minimum light curves of the sources shown in \cref{sec:results} uses this cleaned data.
{\color{red}
\textbf{Marion: add a short summary of the evaluation methods applied to manually clean the data?}

Please reference your table of sources here \cref{app:sources}
}

This is however a very time consuming process.
Iterative outlier rejection schemes that systematically smooths and detrends the time series data of an individual channel can be employed to recursively identify and remove outliers greater than 3$\sigma$.
Such an automated outlier rejection scheme is suggested in \cref{app:cleaning}.
The algorithm implementation was tested against the manually corrected data of the sources.
Reducing the need for manual removal to a minimal intervention.
