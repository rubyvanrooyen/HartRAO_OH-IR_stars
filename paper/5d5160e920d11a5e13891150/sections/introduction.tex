\section{Introduction}
OH/IR stars are late-type stars in the AGB phase of stellar evolution, undergoing substantial mass loss and enshrouded in a thick, expanding shell of dust
\citep[][review]{WilBar68,RJC89}.
The central star heats the dust which re-radiates in the IR.
OH is formed in the outer regions of the dust shell through photo-dissociation of water molecules by UV light.
The thickness of the dust shell, together with the expansion, results in long column densities of OH molecules travelling at velocities with a narrow range of dispersion.
This velocity coherence amongst the OH molecules allows for masing in the dust shell pumped by far-IR photons.
The masers have a characteristic double-peaked profile at 1612 MHz
\citep[][review]{WilBar72,RJC89}
with the most intense radiation arising from the near- and far-sides of the dust shell along the line of sight to the observer.

The central star pulsates resulting in variable heating of the dust and hence IR variations which follow the fluctuations of the central star.
OH maser variations follow the IR variations.
Monitoring the fluctuations in the OH maser radiation allows the phase lag between radiation coming from the front and back of the dust shell to be measured.
This yields the physical size of the OH shell around the star.
The angular diameter of the shell can be measured using interferometry and by combining the two measurements an independent estimate of the distance to the star can be obtained.
The distance measurement is only valid if the OH masing shell obeys the thin shell model
\citep[][review]{ReidMMJS77,RJC89},
ie. it is thin relative to the size of the dust shell and is spherically symmetric.

Determining phase lags from real data appears simple in principle but is non-trivial in practice
Several papers addressing the problem of determining phase lags from lensed quasars highlight some of the difficulties encountered in finding suitable methods
\citep[eg.][]{PhLGil-M02,PhLOscoz01,PhLPelt96,PhLPress92,EK88}.
Standard analytical techniques are appropriate where underlying functions are continuous, stationary, stochastic and regularly sampled
\citep{Welsh99,EK88}.
Real astronomical data rarely fit these criteria.
The \mbox{HartRAO} OH time series, as is generally the case, are continuous and stochastic, but are neither stationary nor regularly sampled.

To determine the phase lag between radiation from the front and back of the OH shell an estimate of the frequency and period of each signals must be obtained.
The HartRAO monitoring data are frequently sampled, making the data well suited for periodogram calculation using the Lomb-Scargle periodogram.
The Lomb-Scargle periodogram is a well-known algorithm for detecting and characterising periodic signals in unevenly sampled data
\citep{2018ApJS..236...16V}.

The HartRAO OH-IR monitoring data will now be publicly available.
The aim of this work is to present a sample of the monitoring data, along with processing methods for phase-lag calculation.

The observational data used in the analysis is described in \cref{sec:observations}.
Data analysis methods are explained in \cref{sec:analysis}.
We present our results on a few selected sources in \cref{sec:results}.
{\color {red}
TBD: \cref{sec:irdata} IR data
}
Lastly, \cref{sec:discussion} present a short discussion of our results.
