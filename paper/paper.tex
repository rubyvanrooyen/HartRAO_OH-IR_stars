\documentclass[usenatbib,usegraphicx]{mn2e}
\usepackage{times}
\hyphenation{Hart-RAO}

\begin{document}

\title{Phase Lag Determination in OH/IR stars using the Discrete Correlation
Function}

\author[M. E. West et al.]
{M. E. West$^{1}$\thanks{E-mail: marion@hartrao.ac.za}, M. J. 
Gaylard$^{1}$, D. J. van der Walt$^{2}$ and R. J. Cohen$^{3}$
\\ $^1$        Hartebeesthoek Radio Astronomy Observatory, PO Box 443,
Krugersdorp, 1740, South Africa
\\ $^2$       Space Research Unit, Physics Department, Potchefstroom
University for CHE, Private Bag X6001, Potchefstroom, 2520, South Africa
}

\date{Accepted 200?;      Received 200?;      in original form 200?}

\pagerange{\pageref{firstpage}--\pageref{lastpage}}
\pubyear{200?}

\maketitle

 \begin{abstract}
Write abstract here...
\end{abstract}

\begin{keywords}
masers -- stars: OH/IR -- stars: AGB -- ...
\end{keywords}

\label{firstpage}

\section{Introduction}

OH/IR stars are late-type stars in the AGB phase of stellar evolution,
undergoing substantial mass loss and enshrouded in a thick, expanding shell
of dust \citep[][review]{WilBar68,RJC89}.  The central star heats the dust
which re-radiates in the IR.  OH is formed in the outer regions of the dust
shell through photo-dissociation of water molecules by UV light.  The
thickness of the dust shell, together with the expansion, results in long
column densities of OH molecules travelling at velocities with a narrow
range of dispersion.  This velocity coherence amongst the OH molecules
allows for masing in the dust shell pumped by far-IR photons.  The masers
have a characteristic double-peaked profile at 1612 MHz
\citep[][review]{WilBar72,RJC89} with the most intense radiation arising
from the near- and far-sides of the dust shell along the line of sight to
the observer.

The central star pulsates resulting in variable heating of the dust and
hence IR variations which follow the fluctuations of the central star.  OH
maser variations follow the IR variations.  Monitoring the fluctuations in
the OH maser radiation allows the phase lag between radiation coming from
the front and back of the dust shell to be measured.  This yields the
physical size of the OH shell around the star.  The angular diameter of the
shell can be measured using interferometry and by combining the two
measurements an independent estimate of the distance to the star can be
obtained.  The distance measurement is only valid if the OH masing shell
obeys the thin shell model \citep[][review]{ReidMMJS77,RJC89}, ie. it is
thin relative to the size of the dust shell and is spherically symmetric.

Determining phase lags from real data appears simple in principle but is
non-trivial in practice.  Several papers addressing the problem of
determining phase lags from lensed quasars highlight some of the
difficulties encountered in finding suitable methods
\citep[eg.][]{PhLGil-M02,PhLOscoz01,PhLPelt96,PhLPress92,EK88}. 
Standard analytical techniques are appropriate where underlying functions
are continuous, stationary, stochastic and regularly sampled
\citep{Welsh99,EK88}.  Real astronomical data rarely fit these criteria. 
The HartRAO OH time series, as is generally the case, are continuous and
stochastic, but are neither stationary nor regularly sampled.

Any method selected for determining the phase lag between radiation from the
front and back of the OH shell must be able to cope with the problems posed
by the data.  Firstly it must either interpolate the data to achieve regular
sampling or be robust enough to cope with irregular sampling \citep{EK88}. 
The OH time series appear superficially sinusoidal and interpolations using
periodic functions such as cosines and asymmetric cosines appear attractive
and have been used in the past \citep[eg.][]{SchSherWin78,HerHab85}. 
However these are not consistent with the quasi-periodic, variable amplitude
light curves actually seen in closely sampled OH/IR light curves.  Thus the
response of any method to non-stationary features needs to be understood.

Several methods of determining phase lags have been developed.  Early
attempts sampled 1612--MHz emission at intervals of a few months over a 2--3
year period and used simple sinusoidal fits to the data
\citep{SchSherWin78,Jewell79,JewWebSny80}.  The monitoring was too short to
model the time series accurately and little more than general confirmation
of theoretical predictions for shell sizes was achieved.  \citet{Herman83}
monitored 1612--MHz OH emission from a set of stars monthly for 4--6 years. 
Coverage was sufficient to show a steeper rise than fall in the time series. 
Phase lags were determined by fitting asymmetric cosines to the time series. 
The data of \citet{HerHab85} were extended to a 10--year span and linear
interpolation and least squares fitting was used for the initial phase lag
estimate.  Errors were estimated using Monte Carlo simulations modelled on
asymmetric cosines fitted to the time series
\citep{vLvdHvS90,vanLangevelde92}.

\citet{Shepherd91} created a template spectrum which modelled the profile
of the source in the absence of variations.  A template time series modelled
on the time series of the strongest spectral peak was also generated.  Data
for any channel could be modelled using a Fourier series evaluation, based
on the assumption that the variations in time series were the same for all
channels apart from an amplitude scaling factor. 
Lags were determined using trial lags in the model spectra and minimising
$\chi^2$ between trial lags the OH time series.  

\citet{West92} interpolated the data using spline fits.  This allowed for
the variations in period and amplitude which are clearly seen in the HartRAO
data.  A least squares minimisation technique was used to determine phase
lags across the shell.  In both the above cases the data used were from the
HartRAO monitoring programme.

All the above methods either fitted analytic functions to the data
\citep[eg.][]{Herman83,HerHab85} or interpolated the data in some other
manner \citep[eg.][]{vLvdHvS90,Shepherd91,West92}.  The analytic functions
forced periods and amplitudes to remain constant over time, subtly altering
data before phase lag determination.  Other types of interpolation gave rise
to a certain amount of `invention' of data \citep{EK88} as the form of the
variations was modelled imperfectly.  Thus all these methods introduced
systematic errors into the time series and phase lags that were difficult to
quantify.

The HartRAO monitoring data are frequently sampled and analytic periodic
functions cannot be used to model the data without sacrificing the detail
gained thereby.  The Discrete Correlation Function (DCF)
method of \citet{EK88}, hereafter referred to as EK88, cross-correlates the
actual data without interpolation, which makes this method particularly
suitable, and it is mathematically relatively simple to implement.  Other
methods used for determining phase lags, eg. from lensed quasars
\citep[eg.][]{PhLPelt96,PhLPress92}, are more complex and some also involve
reconstructing the data.

The aim of this work was to test the DCF method of EK88 for application to
the HartRAO OH data.

The observations data reduction are described in Section 2.  Section 3
explains the selection of sources for testing the method.  In Section 4 the
application of the DCF method to the HartRAO data is briefly described.  In
Section 5 the testing of the DCF method in the context of the OH/IR data is
covered.  In Section 6 the DCF method is applied to the data from two
OH/IRs.  Section 7 deals with the derivation of physically useful parameters
from the results and in Section 8 the efficacy of the DCF method is evaluated.


\section{Observations}

A set of 20 OH/IR stars was selected for monitoring in the ground state OH
lines using the 26--m telescope at Hartebeesthoek \citep{West98}.  They
were monitored from late 1985 until early 1996.  
Initially observations were made
every two weeks, but the rapid increases in brightness
for some of the sources were found to be undersampled and as a result the
observing frequency was increased to once a week in mid 1993.  
GLONASS interference
affected the data
\citep{ComWestGay94} and this eventually led to the termination of the
programme.

The HartRAO antenna had an aperture efficiency of 0.51 when the subreflector
was tilted for observing with the off-axis 18cm feed horn.  The beamwidth
was $0.5^\circ$ and the pointing accuracy was $30''$.  The spectrometer 
accepted only one polarization, and left circular
polarization was normally used.  The zenith system temperature was typically
40--50 K.  A correlator bandwidth of 0.64--MHz, covered by 256
channels, provided a velocity resolution of 0.45 km s$^{-1}$.  Most sources
were observed at 1612--MHz, although some were also observed at 1665-- and
1667--MHz.  The {\em rms\/} noise on a single fifteen minute
spectrum was 0.6 Jy.  Standard continuum sources
\citep{Ott94} and the 1667--MHz OH absorption towards W12 were used for
calibration.

Baseline curvature in the spectra was removed by subtracting low-order
polynomials fitted to signal-free segments. Atmospheric absorption was
corrected using a formula obtained from the Jodrell Bank Observatory.  Data
affected by rain and the proximity of the Sun were discarded.  After
correction, spectra taken in any given 24--hour period were averaged.  Time
series were created for all channels with a useful signal to noise ratio.

Interference from GLONASS satellites increased the noise level in the
spectra, suppressed the maser lines, affected the system temperature
calibration and introduced excessive curvature into the spectrum baselines
\citep{ComWestGay94}.  As a test, one source was corrected manually
\citep{West98}, but this proved too time consuming.  It was replaced by
automatic outlier rejection in the time series of the individual channels, 
in which cubic splines were fitted to the data
and outliers greater than 3$\sigma$ were recursively removed.


\section{Source Selection}


\begin{figure}
\resizebox{\hsize}{!}{\includegraphics[clip,angle=270]{oh357sp.ps}}
\caption{The 1612 MHz OH spectrum of OH357.1--1.3 at maximum and
minimum light.}
\label{fig:oh357sp}
\end{figure}

\begin{figure}
\resizebox{\hsize}{!}{\includegraphics[clip,angle=270]{wxpscsp.ps}}
\caption{The 1612 MHz OH spectrum of WX Psc at maximum and minimum
light.}
\label{fig:wxpscsp}
\end{figure}


Two data sets were selected to test the application of the DCF: a strong
source with a period of six years and 1.5 cycles of data (OH357.3--1.3) and
a weaker source with a period of two years and seven cycles of data (WX
Psc).  Their spectra are shown in Figures~\ref{fig:oh357sp} \&
\ref{fig:wxpscsp}.  OH357.3--1.3 has a profile resembling the canonical `U' 
shape of a spherical-shell OH/IR star, and its emission is detectable at all
points between the two peaks.  By contrast, WX Psc has narrow peaks with no
noticeable emission in between.
The time series of selected channels, including the brightest blue-- and
red--shifted peaks (channels 5 and 54 respectively) and the weakest channel
selected for OH357.3--1.3 (channel 30) are shown in
Figure~\ref{fig:oh357tsp}.  Each time series contained about 260 samples
after outlier removal, providing $\sim170$ samples per cycle.


\begin{figure*}
\resizebox{\hsize}{!}{\includegraphics[clip,angle=0]{oh357tsp.ps}}
\caption{Selected time series from OH357.3-1.3: the velocity of
each channel, with the channel number in parentheses, is shown at top left.}
\label{fig:oh357tsp}
\end{figure*}


The time series of all the channels of WX Psc with a discernible signal are
shown in Figure~\ref{fig:wxtsalp}.  Cycles are numbered for future reference
in the graph for channel 1.  The brightest blue-- and red--shifted peaks are
represented by channels 2 and 3 and channels 11 and 12 respectively.  There
were 330 samples across the whole time series.  For cycles 1 -- 4.5,
observed at two-week intervals, there were typically $\sim45$ samples per
cycle, albeit with significant gaps in cycles 1 and 2.  The remaining one
and a half cycles had double the number of samples per cycle.


\begin{figure*}
\resizebox{\hsize}{!}{\includegraphics[clip,angle=0]{wxtsalp.ps}}
\caption{Time series for WX Psc channels: 
the velocity of the channel, with the channel number in
parentheses, is shown at top left.}
\label{fig:wxtsalp}
\end{figure*}



\section{Application of the DCF Method}

The DCF method of EK88 is based on a convolution function.  It is applicable
in the case of a varying input function which drives an output function.  In
this case, the pulsation-induced varying luminosity of the central star
provides the input function, while the output function is the variation in
the radiatively pumped OH masers in the circumstellar dust shell.

The equations defining the DCF are summarised below.  
For two discrete data trains $a_{i}$ and $b_{j}$ the set of unbinned
discrete correlations is collected:

\begin{equation}
UDCF_{ij} = \frac{(a_{i}-\bar{a})(b_{j}-\bar{b})}
            {\sqrt{(\sigma^{2}_{a}-e^{2}_{a})(\sigma^{2}_{b}-e^{2}_{b})}} 
\end{equation}
for all measured pairs $(a_{i},b_{j})$, each of which is associated with the
pairwise lag $\Delta t_{ij} = t_{j} - t_{i}$, where $t_{i}$ and $t_{j}$ are
the times at which the measurements were taken.  For the OH monitoring data,
$a_{i}$ and $b_{j}$ are the time-dependent flux densities of any two
channels in the OH spectrum.  The mean values are $\bar{a}, \bar{b}$,
the standard deviations from the means are $\sigma_{a}, \sigma_{b}$, and
$e_{a}, e_{b}$ are the associated measurement errors.

The result is then binned in time to enable the directly useful function
$DCF(\tau)$ to be measured.  Averaging over $M$ pairs for which $\tau-
\frac{\Delta\tau}{2} \leq \Delta\tau_{ij} < \tau+\frac{\Delta\tau}{2}$
gives:

\begin{equation}
DCF(\tau) = \frac{1}{M}UDCF_{ij}
\end{equation}

The $DCF(\tau)$ is undefined for a bin with no points.  There is a trade-off
between ensuring high accuracy in the mean for each bin and getting good
resolution in the cross-correlation curve.  Since there is no interpolation,
the DCF is only defined for lags for which measured data exist and the
function can only be evaluated at a discrete set of sample points.  Thus a
mathematical function needs to be fitted across the maximum of the DCF in
order to determine its position accurately \citep[eg.][]{Lehar92}.  EK88
noted that to avoid a spurious maximum at zero lag arising from correlated
errors, points in the time series where the pairwise lag $\Delta t_{ij} =
t_{j} - t_{i} = 0$ should be excluded from the DCF calculation.

                  
The DCF method is relatively insensitive to the size of bin used in the DCF. 
In implementing the DCF, tests indicated that a bin size equal to 
$(0.5 - 1)\times$ the mean sampling interval was appropriate.

For determination of the maximum of the DCF, a time range about zero lag
equal to half the period was used.  With this relatively large range, a
fourth order polynomial was needed to provide a good fit for determining the
maximum.
 

Monte Carlo simulations were used to estimate the error in the phase lag.
The data for the simulations were created as shown in
Table~\ref{tbl:tstplerr}.  One hundred simulations were done for each
non-redundant channel pair used in determining the phase lag.  The error for
each channel pair was estimated using the standard deviation of the mean lag
taken from the Monte Carlo simulations.  The final error in the phase lag
was determined by taking a weighted average of the errors in the phase lags
from each channel pair.

\begin{table}
%\begin{center}
%\begin{minipage}{140mm}

\caption{Characteristics of the function used for estimating errors in the
phase lag.}

\label{tbl:tstplerr}

\begin{tabular}{|l|l|}

\hline
\multicolumn{2}{c}{Characteristics of Function} \\

\hline

Function & Spline fit to real OH time series. \\
Sampling Interval & As for real OH time series. \\
Noise & Constructed from noise in real OH time series. \\
Asymmetry & Implicitly modelled by spline fits to\\
 & real OH time series. \\
Modulations & Implicitly modelled by spline fits to\\
 & real OH time series. \\

\hline

\end{tabular}
%\end{centre}
%\end{minipage}
\end{table}




\section{DCF Testing with Synthetic Data}

\subsection{Creation of the synthetic time series}

In order to determine which characteristics of the OH time series might give
rise to a systematic bias in the calculated phase lags, synthetic data of
similar characteristics but known phase lags were created.

The quasi-periodic, asymmetric nature of the light curves were modelled by
an asymmetric cosine function \citep{DavEtoLS96}:

\begin{equation} 
s(t) = \frac{b\cos (\omega t + \phi)}{1 - f\sin (\omega t +\phi)} 
\end{equation} 
where $s(t)$ is the asymmetric cosine function and $b$ is its amplitude, 
$\omega$ is $\frac{2 \pi}{T}$, where $T$ is the period of the OH variations, 
$t$ is time, $\phi$ is the phase lag and $f$ is related to the asymmetry 
factor $f_{0}$, which lies between zero and one, and is defined as: 

\begin{equation} 
f = \sin (\pi [f_{0} - \frac{1}{2}]) 
\end{equation} 
where, for $f_{0} = 0.5$, the function is symmetric.

OH/IR stars show variations in amplitude from cycle to cycle.  For
simplicity, linear modulations were chosen to model these. Both regular and
irregular sampling were tested.  The latter was simulated using a gaussian
random number generator.  Noise was modelled using a gaussian random number
generator and tests included data with and without noise.

Combining these factors, the equation used to model the data was:

\begin{equation}
F_{i}(t) = \delta_{min}(t) + 
           \frac{\delta_{max}(t)\cos (\omega t+\phi)}
                {1-f\sin (\omega t+\phi)} +
           \frac{\xi(t)}{(1-f)} +
           \frac{\zeta(t)_{t=t_{1}}^{t=t_{2}}}{(1-f)}      
\end{equation}
where $F_{i}(t)$ is the flux density in the $ith$ channel, $t$ is the time
at which the signal was sampled, $\delta_{min}(t)$ is the linear variation
applied to the minima, $\delta_{max}(t)$ is the linear variation applied to
the maxima, $\xi(t)$ is the noise in the data and $\zeta(t)$ are the
zero-lag features applied between times $t_{1}$ and $t_{2}$ .  The factor of
$(1-f)$ is applied to $\xi(t)$ and $\zeta(t)$ to keep the signal to noise
ratio and the zero-lag features, respectively, constant for varying values
of $f$.


\subsection{Testing the characteristics of the OH Time Series}


\begin{table}
%\begin{center}
%\begin{minipage}{140mm}

\caption{Basic function used to model effects in the OH time
series.} 
\label{tbl:tsttschar}

\begin{tabular}{|l|l|}

\hline
\multicolumn{2}{c}{Characteristics of Function} \\

\hline

Function & Cosine \\
Lags & 10$^\circ$ -- 360$^\circ$ in steps of 10$^\circ$ \\
Sampling Interval & 300 points / cycle = 1.2$^\circ$/interval \\

\hline

\end{tabular}
%\end{centre}
%\end{minipage}
\end{table}



\begin{table}
%\begin{center}
%\begin{minipage}{140mm}

\caption{Parameter settings for modelling effects in the OH
time series.}
\label{tbl:parmvartschar}

\begin{tabular}{|l|l|}

\hline
\multicolumn{2}{c}{Characteristics of Function} \\

\hline

Number of Cycles & 1, 2, 4, 8 cycles \\
Asymmetry & f = 0.125, 0.25, 0.375 \\
Linear Trend & 0.5, 0.25, $-0.5\times$ p-p amplitude \\
Sampling Interval & random perturbations with $\sigma$ = \\
 & 0.1, 0.2, 0.3 of sampling interval \\
Noise & 0.05, 0.5$\times$ p-p amplitude \\

\hline

\end{tabular}
%\end{centre}
%\end{minipage}
\end{table}


\subsubsection{Number of cycles of data and start phases of time series}

Several of the sources in the HartRAO OH monitoring programme had fewer than
two cycles of data, but cross-correlation functions are derived on the
assumption of an infinite time series.  \citet{Welsh99} and \citet{Vio01}
have shown that time series with few cycles of data can severely affect the
accurate determination of phase lags.  

Thus the first test was of the bias, 
i.e. the output phase lag minus the input phase lag, 
as a function of start phase and input phase lag, for one cycle of data.
Synthetic data were created using equation (6) and as shown in
Table~\ref{tbl:tsttschar}.  While
lags of up to a full period are not encountered in OH/IR stars, they 
may occur elsewhere.

The results for a single cycle are shown in Figure~\ref{fig:tohtsch}a.  The
bias is clearly dependent on both the input lag and the start phase, and is
symmetric about 180$^\circ$.  For the two test OH/IR stars, the actual lags
between channel pairs are up to 34$^\circ$ for OH357.3--1.3 (maximum lag 200
days over a period of $\sim2100$ days, between extreme outlying channels)
and up to 26$^\circ$ for WX Psc (maximum lag of 47 days, period $\sim650$
days, between channels with extremely weak signal to noise ratios).  It can
be seen that this can produce a bias of $\sim10^\circ$, the direction
depending on the start phase of the observations. Hence for OH/IRs, a
potentially significant systematic error is possible if only a single cycle
of data is available.


\begin{figure}
\resizebox{\hsize}{!}{\includegraphics[clip,angle=0]{tohtsch.ps}}
\caption{Results of tests for (a) a single cycle of pure cosine data with
different start phases, (b) one to eight cycles of pure cosine data with
different start phases (the start phases shown are those giving the largest
positive and negative biases), (c) a single cycle of data with an asymmetry
factor of 0.375 and different start phases (the input function is shown at
bottom left) and (d) a single cycle of data with positive variations in
amplitude equal to half the amplitude and different start phases (the input
function is shown at bottom left).}
\label{fig:tohtsch}
\end{figure}


We expect increasing the number of cycles to decrease the bias, and
this was confirmed by the tests.  
The effects of bias decreased markedly as the number of cycles 
was increased (Table~\ref{tbl:parmvartschar}), 
as shown in Figure~\ref{fig:tohtsch}b.  
The output bias typically halved every time the number of cycles doubled,
and was reduced to ranging between
$-0.8^\circ$ and $+1.7^\circ$ for eight cycles.  
The pattern of the bias as a function of input
lag and start phase remained similar to that for a single cycle as the
number of cycles increased.


\subsubsection{Asymmetry}

Asymmetry was introduced as indicated in Table~\ref{tbl:parmvartschar}.  The
inset in Figure~\ref{fig:tohtsch}c shows the light curve for an asymmetry
factor f of 0.375.  Comparison with with Figure~\ref{fig:tohtsch}a shows
systematic alterations in the bias curves.  The introduction of asymmetric
light curves also destroyed the symmetry about $180^\circ$ noted for the
pure cosine waveform and changed the amplitude of the bias.

\subsubsection{Non-Stationarity}

The introduction of non-stationarity as a linear trend
(Table~\ref{tbl:parmvartschar}) is shown in \ref{fig:tohtsch}d.  The inset
shows the effect on the light curve of a modulation of peak-to-peak
amplitude 0.25.  The pure cosine waveform is deformed in a manner similar to
the asymmetry, but in this example in the opposite sense.  Not surprisingly,
the bias curves are again affected, but in the opposite sense to the
asymmetry test.

\citet{Welsh99} also investigated the effects of modulations on AGN time series
and recommended detrending the data as a
solution to the problems of bias introduced by non-stationary behaviour. 
While this is an alternative way of dealing with modulations
in amplitude, it involves altering the data prior to determining the lag,
an approach which was deliberately avoided here.

\subsubsection{Zero Lag Features}

Features with zero lag were added in accordance with equation (6), modelled
on features seen in the spectra from OH357.3-1.3.  However this had no
effect on the bias.  This was not unexpected as, in accordance with
recommendations by EK88, pairs of points in the time series with zero lag
between them were excluded from the DCF calculation.

\subsubsection{Irregular Sampling and Noise}

When irregular sampling and noise were introduced
(Table~\ref{tbl:parmvartschar}), the bias became randomised in proportion to
the degree of irregular sampling and noise introduced into the time series. 
This is consistent with the results of \citet{Welsh99} for AGNs.



\section{Applying the DCF to the OH Time Series}


\subsection{Applying the DCF Method}

Phase lags were initially determined for all non-redundant pairs of
channels, with errors estimated by Monte Carlo simulation (Section 4).  Next
the characteristics of each channel were scrutinised and channels were
selectively rejected in stages according firstly to whether their behaviour
was consistent with the spherical shell model and secondly on their signal
to noise ratio and how this impacted on the weighted mean phase lag. Each
estimate of the lag across the OH shell was calculated using a weighted mean
of the lag per channel from all channel pairs selected as described above,
multiplied by the number of channels taken as defining the outer edges of
the (assumed spherical) OH shell.


\subsection{Phase Lag for OH357.3--1.3}

Of the 56 channels selected, 30 were taken from the blue--shifted side of
the spectrum, with velocities ranging from $-41.820$ to $-28.339$ kms$^{-1}$
(channels 1 to 30), while 26 were selected from the red--shifted side of the
spectrum, with velocities ranging from $-12.998$ to $-1.376$ kms$^{-1}$
(channels 31 to 56).  The time series data for OH357.3--1.3 were good: they
were well sampled, there were no large gaps in the sampling and the signal
to noise ratio was generally high (Figure~\ref{fig:oh357tsp}).  However, with 
only 1.5 cycles of data, the effects of bias could
be not be neglected. 

To identify channels fitting the spherical shell model, and those which are
outliers, the phase lags of all channels were plotted, using channel 5 as a
phase reference (Figure~\ref{fig:oh357l5s}). Some channels depart
significantly from the linear regression characteristic of a spherical
shell, particularly channels 1, 2, 3, 50, 53, 55 and 56.

To further facilitate outlier identification, the weighted means of the lags
per channel were plotted against the weighted standard deviations of the
lags per channel (Figure~\ref{fig:oh357acn}).  The majority of channels
cluster above channel 54, while the outliers spread out to the sides.  In
particular, channels 1, 2, 3, 50, 55 \& 56 show larger lags per channel and
a linear trend which is indicative OH beyond the main shell with linear
outflow.  Channels 15, 35, 49 \& 53 have smaller lags per channel and are
more randomly distributed, indicating clumps of OH closer to the star than
the main shell.  From Figure~\ref{fig:oh357l5s} it is clear that channels
1--3 lie on the near side of the shell, while channels 50, 55 \& 56 lie on
the far side.


\begin{figure}
\resizebox{\hsize}{!}{\includegraphics[clip,angle=270]{oh357l5s.ps}}
\caption{Phase lags from all channels selected for OH357.3--1.3, using
channel 5 as the reference channel.  A linear regression is fitted through
lags from all channels, excluding those from outlying channels.}
\label{fig:oh357l5s}
\end{figure}

\begin{figure}
\resizebox{\hsize}{!}{\includegraphics[clip,angle=270]{oh357acn.ps}}
\caption{Phase lags per channel for all channels selected for OH357.3-1.3,
showing the difference between lags from standard channels and outliers.}
\label{fig:oh357acn}
\end{figure}


Referring to the time series shown in Figure~\ref{fig:oh357tsp}, channels 5,
30, 40 \& 54 are representative of the channels fitting the spherical shell.
The other time series represent the identified outliers.  Their light curves
are clearly different. This is most strikingly seen in the three adjacent
channels 53, 54 and 55. Channel 54 is the brightest red-shifted peak. 
Channels 55 and 56 lag significantly behind channel 54: the second maximum
had not yet been reached when observing ended.  By contrast, channel 53
leads all the other channels on the red-shifted side of the spectrum. While
all three channels are adjacent in terms of velocity, the two atypical
channels are evidently spatially separated from the main OH shell.

Defining the outer edges of the shell for OH357.3--1.3 as the most extreme
blue- and red-shifted channels not found to be outliers (channels 4 \& 54,
at $-40.425$kms$^{-1}$ \& $2.306$kms$^{-1}$ respectively) the phase lag
including all 56 channels is $43\pm35$ days. On the basis of
Figures~\ref{fig:oh357tsp}, \ref{fig:oh357l5s} and \ref{fig:oh357acn},
channels 1, 2, 3, 15, 35, 49, 50, 53, 55 \& 56 were classed as outliers and
removed from the estimate of the mean phase lag.  The phase lag is then
$37\pm16$ days, so that the error has more than halved by removing only
$\sim20\%$ of the channels.


The effect of channels with low signal to noise ratios was also
investigated.  Removal of the six non-outlying channels with the highest
standard deviations (26 to 31) changed the phase lag from $37\pm16$ days to
$38\pm14$ days.  Removal of a further 15 channels with high standard
deviations (17 to 25, 32 to 34 \& 37 to 39), resulted in a phase lag of
$38\pm11$ days. The removal of both these sets of channels only altered the
phase lag by $\sim3\%$, whereas the removal of oultiers had changed the
phase lag by $\sim15\%$.  Channels with a low signal to noise ratio
increased the error in the phase lag, without significantly affecting the
value of the lag.  Finally, using only the four brightest channels the phase
lag was $37\pm5$ days.

The initial lag obtained when outliers were included agrees within the
errors given here with the lag of 50 days obtained by \citet{West92}.  At
that time the authors were unaware that some of the channels represented
outlying OH and these channels were used in the calculation of the final
lag.  The error analysis for the earlier method was not robust and hence the
error estimate for those data was extremely optimistic.


\subsection{Estimating the Bias for OH357.3--1.3}

Model data were created using equation (6), with the parameters period =
2075$\pm25$ days, number of cycles = 1.51$\pm0.02$, asymmetry factor =
$0.385\pm0.01$, start phase = $920\pm50$ days.  The increases in amplitude
from minimum to minimum and maximum to maximum were measured and applied. 
For variations in the minima the fractional change was $0.253\pm0.02$ per
cycle and for the maximum it was $0.246\pm0.05$ per cycle.

The measured phase lag is strongly weighted by the values for the main
peaks, so it is appropriate to correct for the bias for a lag of 37 days,
which the simulation indicates is $\sim-21\%$, or --10 days.  The corrected
phase lag is then 47 days.

Results for the OH which lies outside the main shell are shown in
Table~\ref{tbl:outliers}.  The bias has been calculated for each of the
outlying channels listed.  The values were not included in the calculations
of the distance of the outlying OH from the star, as they are small enough
not to make a significant difference.


\begin{table*}
%\begin{center}
%\begin{minipage}{140mm}

\caption{Outlying OH in OH357.3--1.3.}
\label{tbl:outliers}

\begin{tabular}{|c|c|r|r|r|r|r|}

\hline

\multicolumn{2}{c}{Channel} &
\multicolumn{2}{c}{Phase Lag (days)} &
\multicolumn{1}{c}{Bias} &
\multicolumn{2}{c}{Radius} \\

\multicolumn{1}{c}{Number} &
\multicolumn{1}{c}{Reference} &
\multicolumn{1}{c}{Reference Channel} &
\multicolumn{1}{c}{Star} &
\multicolumn{1}{c}{(days)} &
\multicolumn{1}{c}{$\times10^{16}$cm} &
\multicolumn{1}{c}{$\times10^{3}$AU} \\


\hline

1 & 4 & $-24\pm3$ & $-41\pm4$ & 3 & 
 $10.62\pm0.1$ & $7.10\pm0.7$ \\
2 & 4 & $-12\pm3$ & $-29\pm4$ & 1 &
 $7.52\pm0.1$ & $5.02\pm0.7$ \\
3 & 4 & $-20\pm3$ & $-37\pm4$ & 3 &
 $9.59\pm0.1$ & $6.41\pm0.7$ \\
50 & 54 & $28\pm3$ & $48\pm4$ & 2 &
 $11.66\pm0.1$ & $7.80\pm0.7$ \\
55 & 54 & $52\pm3$ & $69\pm4$ & 2 &
 $17.89\pm0.1$ & $11.96\pm0.7$ \\
56 & 54 & $104\pm6$ & $121\pm6$ & -4 &
 $31.36\pm0.2$ & $20.97\pm1.0$ \\

\hline

\end{tabular}
%\end{centre}
%\end{minipage}
\end{table*}


\subsection{Phase Lag for WX Psc}


The spectrum of WX Psc provides only a comparatively small number of
channels from which to determine the phase lag (Figure~\ref{fig:wxtsalp}).
The data comprised six cycles, of which the cycles 1 and 2 were of poor
quality, being noisy and with gaps of up to half a cycle
(Figure~\ref{fig:wxtsalp}).  Tests showed that such gaps could seriously
bias the lag.  Cycles 3 -- 6 were generally good: they were
relatively well sampled and the signal to noise ratio was reasonable.  Cycle
6 had the best quality, being sampled at one week intervals. Cycle 7 was
incomplete and noisy and was omitted.

\begin{table*}
%\begin{center}
%\begin{minipage}{140mm}

\caption{Results for WX Psc.}
\label{tbl:reswxpsc}

\begin{tabular}{|l|l|l|l|l|l|}

\hline

\multicolumn{1}{c}{} &
\multicolumn{5}{c}{Phase Lag (days)} \\

\multicolumn{1}{c}{Channels Used} &
\multicolumn{1}{c}{Cycles 3 -- 6} &
\multicolumn{1}{c}{Cycle 3} &
\multicolumn{1}{c}{Cycle 4} &
\multicolumn{1}{c}{Cycle 5} &
\multicolumn{1}{c}{Cycle 6} \\


\hline

All Channels &  $31\pm11$ & $34\pm23$ &
 $32\pm9$ & $36\pm19$ & $31\pm10$ \\
10 Brightest Channels &  $31\pm11$ & $33\pm22$ &
 $32\pm9$ & $36\pm14$ & $31\pm10$ \\
6 Brightest Channels & $32\pm8$ &  $35\pm14$ &
 $30\pm6$ & $31\pm10$ & $32\pm7$ \\
4 Brightest Channels & $29\pm6$ & $30\pm8$ &
 $32\pm6$ & $31\pm2$ & $29\pm5$ \\

\hline

\end{tabular}
%\end{centre}
%\end{minipage}
\end{table*}


In contrast to OH357.3--1.3, there is a predominance of outliers: the
channels do not behave as would be expected for a uniformly expanding
spherical shell, a result consistent with its narrow-peaked spectrum
(Figure~\ref{fig:wxpscsp}). The weighted means of the lags per channel
versus the weighted standard deviations of the lags per channel are shown in
Figure~\ref{fig:fstat4cn} as an alternative visualisation of this.


\begin{figure}
\resizebox{\hsize}{!}{\includegraphics[clip,angle=270]{fstat4cn.ps}}
\caption{Phase lags per channel for all channels selected for WX Psc.}
\label{fig:fstat4cn}
\end{figure}

The effect of omitting channels starting with those with the lowest signal
to noise ratio is shown in Table~\ref{tbl:reswxpsc}. Initially the three
weakest inner channels were omitted (6, 7 \& 8), then the next four weakest
channels (1, 5, 9 \& 13) and finally all but the four brightest channels
were omitted (4 \& 10 omitted, 2, 3, 11 \& 12 retained).  This produces
little change in the phase lag, but the error estimate is typically halved. 
What is not known is how representative the brightest channels are of the
(assumed) spherical shell, but we define the outer edges of the shell as the
blue- and red-shifted channels least resembling outliers
(Figures~\ref{fig:fstat4cn} \& \ref{fig:pldiags}): channels 2 \& 11, at
$-9.521$kms$^{-1}$ \& $26.739$kms$^{-1}$ respectively.
Figure~\ref{fig:pldiags} shows the lags using channel 2 as a reference, with
a linear regression fitted through the six brightest channels.


\begin{figure}
\resizebox{\hsize}{!}{\includegraphics[clip,angle=270]{pldiags.ps}}
\caption{Phase lags from all channels used for WX Psc, with channel 2
as the reference.  A linear regression is fitted through the lags
for the six brightest channels.}
\label{fig:pldiags}
\end{figure}


The data were tested to see if any change in OH shell size could be seen
over the monitoring period. Phase lags were calculated for individual cycles
and compared (Table~\ref{tbl:reswxpsc}).  Within the errors, there is no
evidence for changes in shell size over the four cycles.  The error
estimates indicate that data for the 4th \& 6th cycles are more reliable
than those for the 3rd \& 5th (see also Figure~\ref{fig:fstata4c}).  It
appears that the best result obtained for the 5th cycle, using only the four
brightest channels.  That this is spurious can be clearly seen from
figure~\ref{fig:fstata4c}: results for channels 2 \& 3 and 11\& 12 almost
coincide, thus the small error is the result of small number statistics
rather than superior data.


\begin{figure}
\resizebox{\hsize}{!}{\includegraphics[clip,angle=270]{fstata4c.ps}}
\caption{Phase lags per channel for individual cycles of data for WX Psc. 
Arrows indicate where channels lie off the graph.}
\label{fig:fstata4c}
\end{figure}


\subsection{Estimating the Bias for WX Psx}

The model data used the following parameters from the actual data: period =
645$\pm10$ days, number of cycles = 4.0, asymmetry factor = $0.45\pm0.01$,
start phase = $320\pm15$ days.  For linear modulations, the fractional
change in the maxima value = $0.60\pm0.02$ per cycle and for the minima
= $0.37\pm0.06$ per cycle.

The phase lag for the brightest channels, which dominate the weighted mean
lag estimate, is about 29 days. The simulations show that the bias for this
lag, over 4 cycles, was $\sim-6\%$, or $-2$ days. This implies that best
estimate the lag in WX Psc is $31\pm6$ days. This can be compared with
estimate of $34.0\pm5.6$ days using independent data, obtained by
\citet{vLvdHvS90}.


\section{Derived Parameters for the OH/IR stars}

\subsection{Data sources}

OH shell size estimates are available for both stars, permitting their
distances to be calculated.  As the fluxes for both sources were
concentrated in the IR, Their total fluxes were obtained by integrating
fluxes from NIR observations of the sources made at the South African
Astronomical Observatory (SAAO), fluxes estimated from the Infrared
Astronomical Satellite (IRAS) Low Resolution Spectrograph (LRS) spectrum for
WX Psc \citep{VolkCohen89}, and the IRAS fluxes from the IRAS Point Source
Catalogue (PSC).  Bolometric luminosities could then be calculated for the
sources.


\subsection{Derived Parameters for OH357.3--1.3}

Using the phase lag estimate of $47\pm5$ days the resultant shell radius is
$(6.091\pm0.648)\times 10^{16}$cm, or $4.072\times 10^{3}$AU.  Taking the
angular radius as 3.01$\pm0.14$ arcsec, as determined from MERLIN
\citep{Shepherd91,Shepherd93} the distance to OH357.3--1.3 becomes
$1.346\pm0.14$ kpc.  This can be compared with the distance of $0.99$
determined by \citet{OliWhtlkMar01} using the period-luminosity relation.

The distance combined with the mean integrated infrared flux of 5.0$\times
10^{-10}$Wm$^{-2}$ implies a bolometric luminosity of $-6.3\pm0.7$ for
OH357.3--1.3, which is in agreement with that of $-6.17$ obtained by
\citet{OliWhtlkMar01}.  This corresponds to a luminosity of $(27\pm3)\times
10^{3} L_\odot$.  The bolometric luminosity implies that the mass of the
progenitor star could have ranged from $\sim??-??$ solar masses (e-mail PAW)
\citep{IbenRen83}.  AGB stars with progenitor masses less than eight solar
masses do not ignite carbon in their core and their luminosities can range
from $\sim2 - 70\times 10^{3} L_\odot$ \citep{IbenRen83}.

The channels lying outside the main peaks (1, 2, 3, 50, 55, 56,
Figure~\ref{fig:oh357tsp}) show phase lags that deviate systematically from
the trend of the channels within the main peaks (Figure~\ref{fig:oh357l5s}). 
Their behaviour is consistent with their being emission from OH lying
outside the main shell of OH masers.  The gas has evidently almost reached
terminal velocity, given the small velocity range ($\sim1$kms$^{-1}$) from
the end of the emission to the brightest blue- and red-shifted peaks. The
phase lags of these channels relative to the central star and their
respective radii are shown in Table~\ref{tbl:outliers}.  The OH giving rise
to the maser radiation in these channels lies at distances from about two to
seven times the main shell radius.  The OH for the majority of these
channels lies between two and three times the shell radius from the star. 
The OH represented by channels 55 \& 56 however lies at four and seven times
the shell radius respectively.


\subsection{Derived Parameters for WX Psc}

Assuming the spherical shell model and using a phase lag of $31\pm6$ days,
implies a shell of radius $(4.018\pm0.78)\times 10^{16}$cm, or
$2.69\times 10^3$ AU.  The angular radius of 4.4 arcsec was determined from
the VLA by \citet{BowJS83}.  The distance is then $0.61\pm0.12$ kpc.  This
agrees within the errors with the distance of $0.74\pm0.15$ obtained by
\citet{vLvdHvS90} using the phase lag method and is therefore in agreement
with that of 0.74 from \citet{OliWhtlkMar01} using the period-luminosity
relation.

The mean infrared flux is $7.0\times10^{-10}$Wm$^{-2}$.  At the inferred
distance this leads to a bolometric luminosity of $-5.0\pm1.1$, which is in
agreement with that of $-5.35$ obtained by \citet{OliWhtlkMar01} and
corresponds to a luminosity of $(8\pm2)\times 10^3 L_\odot$. 
\citet{vLvdHvS90} obtained a luminosity of $39.4\times 10^3 L_\odot$ for WX
Psc which appears to be in error.  The bolometric luminosity implies that
the mass of the progenitor star could have ranged from $\sim??-??$ solar
masses (e-mail PAW) \citep{IbenRen83}.


\section{Discussion}

The data used to test the DCF method were well sampled but have a very large
range of signal to noise ratio.  The DCF proved to be a suitable method to
apply to these data.  Synthetic data closely modelled on the real data
confirmed this and permitted critical points to be tested, for example
having data with little more than one cycle, of asymmetric form and changing
significantly from one cycle to the next.  These simulations showed that
bias is a significant factor in data with few cycles.  Monte Carlo
simulations were used to obtain realistic error estimates.

Turning to the data on the two OH/IR stars, for OH357.3-1.3, the phase lag
per channel for most channels is consistent with their lying in a spherical
shell in a symmetric outflow.  Four channels appear to derive from clumps
closer to the central star (15, 35, 49 \& 53), by virtue of their lag per
channel being substantially smaller than the mean.  The OH at the edges of
the spectrum show much larger phase lags than the main shell and are
evidently blowing away at almost terminal velocity at up to seven times the
radius of the main OH shell.

The narrow peaks in the spectrum of WX Psc indicate that it does not have a
thick, uniformly expanding spherical shell and the phase lags per channel
confirm this.

Examination of the data showed that channels with low signal to noise ratios
increased the error in the phase lag estimate without radically affecting
the value of the lag.  Thus it is important to examine the time series data
carefully to determine (i) whether there are any channels which are not part
of the main shell and (ii) what minimum set of channels should be retained
in order to give a reliable estimate of the lag with the smallest reasonable
estimate of the error.

The distances calculated using the phase lags from the DCF method agreed
with results obtained using the period-luminosity relation
\citep{OliWhtlkMar01}.  The distance obtained for WX Psc also agreed with
that of \citet{vLvdHvS90} using independent time series data.  Calculations
of bolometric luminosities showed that these sources have progenitor masses
within the range of those expected for oxygen-rich AGB stars and that the
progenitor mass of the source with the shorter period (WX Psc) is lower than
that of the longer period source (OH357.3-1.3).

Tests showed that while WX Psc was monitored over several cycles, there was
no indication of change in the size of the OH maser shell.  Thus for this
source, while the dust expands outwards, the location of the masers remains
relatively constant.

      
   
\section*{Acknowledgements}

The authors thank staff at HartRAO who
helped with running the observations.

%\clearpage


\bibliographystyle{mn2e}
\bibliography{mn-jour,refs-adsabs.bib}
\label{lastpage}
\end{document}
